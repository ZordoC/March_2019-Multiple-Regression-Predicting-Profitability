\documentclass[]{article}
\usepackage{lmodern}
\usepackage{amssymb,amsmath}
\usepackage{ifxetex,ifluatex}
\usepackage{fixltx2e} % provides \textsubscript
\ifnum 0\ifxetex 1\fi\ifluatex 1\fi=0 % if pdftex
  \usepackage[T1]{fontenc}
  \usepackage[utf8]{inputenc}
\else % if luatex or xelatex
  \ifxetex
    \usepackage{mathspec}
  \else
    \usepackage{fontspec}
  \fi
  \defaultfontfeatures{Ligatures=TeX,Scale=MatchLowercase}
\fi
% use upquote if available, for straight quotes in verbatim environments
\IfFileExists{upquote.sty}{\usepackage{upquote}}{}
% use microtype if available
\IfFileExists{microtype.sty}{%
\usepackage{microtype}
\UseMicrotypeSet[protrusion]{basicmath} % disable protrusion for tt fonts
}{}
\usepackage[margin=1in]{geometry}
\usepackage{hyperref}
\hypersetup{unicode=true,
            pdftitle={Multiple Regression},
            pdfauthor={José Pedro Conceição,Kiko Sánchez , Eloi Cirera},
            pdfborder={0 0 0},
            breaklinks=true}
\urlstyle{same}  % don't use monospace font for urls
\usepackage{color}
\usepackage{fancyvrb}
\newcommand{\VerbBar}{|}
\newcommand{\VERB}{\Verb[commandchars=\\\{\}]}
\DefineVerbatimEnvironment{Highlighting}{Verbatim}{commandchars=\\\{\}}
% Add ',fontsize=\small' for more characters per line
\usepackage{framed}
\definecolor{shadecolor}{RGB}{248,248,248}
\newenvironment{Shaded}{\begin{snugshade}}{\end{snugshade}}
\newcommand{\KeywordTok}[1]{\textcolor[rgb]{0.13,0.29,0.53}{\textbf{#1}}}
\newcommand{\DataTypeTok}[1]{\textcolor[rgb]{0.13,0.29,0.53}{#1}}
\newcommand{\DecValTok}[1]{\textcolor[rgb]{0.00,0.00,0.81}{#1}}
\newcommand{\BaseNTok}[1]{\textcolor[rgb]{0.00,0.00,0.81}{#1}}
\newcommand{\FloatTok}[1]{\textcolor[rgb]{0.00,0.00,0.81}{#1}}
\newcommand{\ConstantTok}[1]{\textcolor[rgb]{0.00,0.00,0.00}{#1}}
\newcommand{\CharTok}[1]{\textcolor[rgb]{0.31,0.60,0.02}{#1}}
\newcommand{\SpecialCharTok}[1]{\textcolor[rgb]{0.00,0.00,0.00}{#1}}
\newcommand{\StringTok}[1]{\textcolor[rgb]{0.31,0.60,0.02}{#1}}
\newcommand{\VerbatimStringTok}[1]{\textcolor[rgb]{0.31,0.60,0.02}{#1}}
\newcommand{\SpecialStringTok}[1]{\textcolor[rgb]{0.31,0.60,0.02}{#1}}
\newcommand{\ImportTok}[1]{#1}
\newcommand{\CommentTok}[1]{\textcolor[rgb]{0.56,0.35,0.01}{\textit{#1}}}
\newcommand{\DocumentationTok}[1]{\textcolor[rgb]{0.56,0.35,0.01}{\textbf{\textit{#1}}}}
\newcommand{\AnnotationTok}[1]{\textcolor[rgb]{0.56,0.35,0.01}{\textbf{\textit{#1}}}}
\newcommand{\CommentVarTok}[1]{\textcolor[rgb]{0.56,0.35,0.01}{\textbf{\textit{#1}}}}
\newcommand{\OtherTok}[1]{\textcolor[rgb]{0.56,0.35,0.01}{#1}}
\newcommand{\FunctionTok}[1]{\textcolor[rgb]{0.00,0.00,0.00}{#1}}
\newcommand{\VariableTok}[1]{\textcolor[rgb]{0.00,0.00,0.00}{#1}}
\newcommand{\ControlFlowTok}[1]{\textcolor[rgb]{0.13,0.29,0.53}{\textbf{#1}}}
\newcommand{\OperatorTok}[1]{\textcolor[rgb]{0.81,0.36,0.00}{\textbf{#1}}}
\newcommand{\BuiltInTok}[1]{#1}
\newcommand{\ExtensionTok}[1]{#1}
\newcommand{\PreprocessorTok}[1]{\textcolor[rgb]{0.56,0.35,0.01}{\textit{#1}}}
\newcommand{\AttributeTok}[1]{\textcolor[rgb]{0.77,0.63,0.00}{#1}}
\newcommand{\RegionMarkerTok}[1]{#1}
\newcommand{\InformationTok}[1]{\textcolor[rgb]{0.56,0.35,0.01}{\textbf{\textit{#1}}}}
\newcommand{\WarningTok}[1]{\textcolor[rgb]{0.56,0.35,0.01}{\textbf{\textit{#1}}}}
\newcommand{\AlertTok}[1]{\textcolor[rgb]{0.94,0.16,0.16}{#1}}
\newcommand{\ErrorTok}[1]{\textcolor[rgb]{0.64,0.00,0.00}{\textbf{#1}}}
\newcommand{\NormalTok}[1]{#1}
\usepackage{longtable,booktabs}
\usepackage{graphicx,grffile}
\makeatletter
\def\maxwidth{\ifdim\Gin@nat@width>\linewidth\linewidth\else\Gin@nat@width\fi}
\def\maxheight{\ifdim\Gin@nat@height>\textheight\textheight\else\Gin@nat@height\fi}
\makeatother
% Scale images if necessary, so that they will not overflow the page
% margins by default, and it is still possible to overwrite the defaults
% using explicit options in \includegraphics[width, height, ...]{}
\setkeys{Gin}{width=\maxwidth,height=\maxheight,keepaspectratio}
\IfFileExists{parskip.sty}{%
\usepackage{parskip}
}{% else
\setlength{\parindent}{0pt}
\setlength{\parskip}{6pt plus 2pt minus 1pt}
}
\setlength{\emergencystretch}{3em}  % prevent overfull lines
\providecommand{\tightlist}{%
  \setlength{\itemsep}{0pt}\setlength{\parskip}{0pt}}
\setcounter{secnumdepth}{0}
% Redefines (sub)paragraphs to behave more like sections
\ifx\paragraph\undefined\else
\let\oldparagraph\paragraph
\renewcommand{\paragraph}[1]{\oldparagraph{#1}\mbox{}}
\fi
\ifx\subparagraph\undefined\else
\let\oldsubparagraph\subparagraph
\renewcommand{\subparagraph}[1]{\oldsubparagraph{#1}\mbox{}}
\fi

%%% Use protect on footnotes to avoid problems with footnotes in titles
\let\rmarkdownfootnote\footnote%
\def\footnote{\protect\rmarkdownfootnote}

%%% Change title format to be more compact
\usepackage{titling}

% Create subtitle command for use in maketitle
\providecommand{\subtitle}[1]{
  \posttitle{
    \begin{center}\large#1\end{center}
    }
}

\setlength{\droptitle}{-2em}

  \title{Multiple Regression}
    \pretitle{\vspace{\droptitle}\centering\huge}
  \posttitle{\par}
    \author{José Pedro Conceição,Kiko Sánchez , Eloi Cirera}
    \preauthor{\centering\large\emph}
  \postauthor{\par}
      \predate{\centering\large\emph}
  \postdate{\par}
    \date{March 25, 2019}


\begin{document}
\maketitle

{
\setcounter{tocdepth}{5}
\tableofcontents
}
\subsection{Executive Summary}\label{executive-summary}

The first Objective was accurately Predicting Sales Volume, this has not
been fully achieved because of the small data set provided to the team,
we did train some models and made some predictions,however, they
accurate and the errors are big, but it was the best information value
we could extract from this sample. Predicting the sales of new Products
using a reduced sample won't be accurate, besides being statistically
unsound. We found out that what actually predicts success in volume are
both, however the best predictor for successe comes from service reviews
with an importance of 100\% according to a random forest algorithm,
followed by a a 50\% importance of 4 star reviews.\\
 Why 4 star reviews and not other reviews ? Well because they had to be
taken out of our model training, they had levels of relationship with
our predictor (Volume) so high that they were biasing the whole model,
making the predictions even more unreliable.For example the 5 star
review had a perfect correlation with the Volume, this means that the
volume would grow at the same rate as a the 5 star reviews increased,
which does not translate into reality. Nonetheless here are our final
predictions.

\begin{longtable}[]{@{}lcr@{}}
\toprule
Number & Volume & Profit\tabularnewline
\midrule
\endhead
Game Console & 1735.3872 & 347.07744\tabularnewline
Game Console & 1735.3872 & 312.36970\tabularnewline
Tablet & 1319.6391 & 118.76752\tabularnewline
Tablet & 1211.3776 & 109.02398\tabularnewline
NoteBook & 877.6156 & 87.76156\tabularnewline
\bottomrule
\end{longtable}

\subsection{Technical Report}\label{technical-report}

\subsubsection{Pre-process}\label{pre-process}

We always start by assesing the importance of each variable, we achieve
this by doing a correlation matrix and training a simple model, followed
by a varImp(), which will give us in percentage the importance of the
variable for the model's prediction.(We need to first use a correlation
matrix to see the correlation values, and take out anything that migh
bias our model, otherwise the ``biased'', features will just appear at
the top of the varImp() output).\\
 We created a function to dummyfy the variables and to check if there
was any NA values (in any attribute), and if they exist, remove them, I
also included a function that removes outliers, and a function to subset
the data into different product types.

\subsubsection{Process functions}\label{process-functions}

\paragraph{Pre-process}\label{pre-process-1}

\begin{Shaded}
\begin{Highlighting}[]
\NormalTok{PPfunction <-}\StringTok{ }\ControlFlowTok{function}\NormalTok{(data) \{}
  
\NormalTok{  N <-}\StringTok{ }\KeywordTok{dummyVars}\NormalTok{(}\StringTok{" ~ ."}\NormalTok{, }\DataTypeTok{data =}\NormalTok{ data)}
  
\NormalTok{  N <-}\StringTok{ }\KeywordTok{data.frame}\NormalTok{(}\KeywordTok{predict}\NormalTok{(N, }\DataTypeTok{newdata =}\NormalTok{ data))}
  
\NormalTok{  N <-}\StringTok{ }\NormalTok{N[,}\KeywordTok{colSums}\NormalTok{(}\KeywordTok{is.na}\NormalTok{(N)) }\OperatorTok{==}\StringTok{ }\DecValTok{0}\NormalTok{] }
  
\NormalTok{  N }
\NormalTok{\}}
\end{Highlighting}
\end{Shaded}

\paragraph{Remove Outliers}\label{remove-outliers}

\begin{Shaded}
\begin{Highlighting}[]
\NormalTok{RmOut <-}\StringTok{ }\ControlFlowTok{function}\NormalTok{(D,V)}

\NormalTok{  \{}
  
\NormalTok{  Out <-}\StringTok{ }\KeywordTok{boxplot}\NormalTok{(D}\OperatorTok{$}\NormalTok{V ,}\DataTypeTok{plot =} \OtherTok{FALSE}\NormalTok{)}\OperatorTok{$}\NormalTok{out}
\NormalTok{  K <-}\StringTok{ }\NormalTok{D[}\OperatorTok{-}\KeywordTok{which}\NormalTok{(D}\OperatorTok{$}\NormalTok{V }\OperatorTok\StringTok{ }\NormalTok{Out),]}
\NormalTok{  K}
  
\NormalTok{\}}
\end{Highlighting}
\end{Shaded}

\paragraph{Sub-set by product types}\label{sub-set-by-product-types}

\begin{Shaded}
\begin{Highlighting}[]
\NormalTok{SubSetDataProductTypes <-}\StringTok{ }\ControlFlowTok{function}\NormalTok{(data,p,}\DataTypeTok{p1 =} \DecValTok{0}\NormalTok{,}\DataTypeTok{p2 =} \DecValTok{0}\NormalTok{ , }\DataTypeTok{p3 =} \DecValTok{0}\NormalTok{ , }\DataTypeTok{p4 =} \DecValTok{0}\NormalTok{)}

\NormalTok{  \{}
  \ControlFlowTok{if}\NormalTok{ ( p1 }\OperatorTok{==}\StringTok{ }\DecValTok{0} \OperatorTok{&&}\StringTok{ }\NormalTok{p2 }\OperatorTok{==}\StringTok{ }\DecValTok{0} \OperatorTok{&&}\StringTok{ }\NormalTok{p3 }\OperatorTok{==}\StringTok{ }\DecValTok{0} \OperatorTok{&&}\StringTok{ }\NormalTok{p4 }\OperatorTok{==}\StringTok{ }\DecValTok{0}\NormalTok{)}
\NormalTok{  \{}
\NormalTok{    Nsub <-}\StringTok{ }\KeywordTok{subset}\NormalTok{(data, data}\OperatorTok{$}\NormalTok{ProductType }\OperatorTok{==}\StringTok{ }\NormalTok{p)}
    
    \KeywordTok{return}\NormalTok{(Nsub)}
\NormalTok{  \}}
  \ControlFlowTok{else} \ControlFlowTok{if}\NormalTok{ (p2 }\OperatorTok{==}\StringTok{ }\DecValTok{0} \OperatorTok{&&}\StringTok{ }\NormalTok{p3 }\OperatorTok{==}\StringTok{ }\DecValTok{0} \OperatorTok{&&}\StringTok{ }\NormalTok{p4 }\OperatorTok{==}\StringTok{ }\DecValTok{0}\NormalTok{)\{}
\NormalTok{    Nsub <-}\StringTok{ }\KeywordTok{subset}\NormalTok{(data, data}\OperatorTok{$}\NormalTok{ProductType }\OperatorTok{==}\StringTok{ }\NormalTok{p)}
    
\NormalTok{    Nsub2 <-}\KeywordTok{subset}\NormalTok{(data, data}\OperatorTok{$}\NormalTok{ProductType }\OperatorTok{==}\StringTok{ }\NormalTok{p1)}
    
\NormalTok{    Nsub2 <-}\StringTok{ }\KeywordTok{rbind}\NormalTok{(Nsub,Nsub2)}
    
    \KeywordTok{return}\NormalTok{(Nsub2)}
\NormalTok{  \}}
  \ControlFlowTok{else}  \ControlFlowTok{if}\NormalTok{ (p3 }\OperatorTok{==}\StringTok{ }\DecValTok{0} \OperatorTok{&&}\StringTok{ }\NormalTok{p4 }\OperatorTok{==}\StringTok{ }\DecValTok{0}\NormalTok{) }
\NormalTok{  \{}
    
\NormalTok{    Nsub <-}\StringTok{ }\KeywordTok{subset}\NormalTok{(data, data}\OperatorTok{$}\NormalTok{ProductType }\OperatorTok{==}\StringTok{ }\NormalTok{p)}
    
\NormalTok{    Nsub2 <-}\KeywordTok{subset}\NormalTok{(data, data}\OperatorTok{$}\NormalTok{ProductType }\OperatorTok{==}\StringTok{ }\NormalTok{p1)}
    
\NormalTok{    Nsub3 <-}\StringTok{ }\KeywordTok{subset}\NormalTok{(data,data}\OperatorTok{$}\NormalTok{ProductType }\OperatorTok{==}\StringTok{ }\NormalTok{p2)}
    
\NormalTok{    Nsub3 <-}\StringTok{ }\KeywordTok{rbind}\NormalTok{(Nsub,Nsub2,Nsub3)}
    
    \KeywordTok{return}\NormalTok{(Nsub3)}
    
\NormalTok{  \}}
  
  \ControlFlowTok{else}  \ControlFlowTok{if}\NormalTok{ (p4 }\OperatorTok{==}\StringTok{ }\DecValTok{0}\NormalTok{)\{}
\NormalTok{    Nsub <-}\StringTok{ }\KeywordTok{subset}\NormalTok{(data, data}\OperatorTok{$}\NormalTok{ProductType }\OperatorTok{==}\StringTok{ }\NormalTok{p)}
    
\NormalTok{    Nsub2 <-}\KeywordTok{subset}\NormalTok{(data, data}\OperatorTok{$}\NormalTok{ProductType }\OperatorTok{==}\StringTok{ }\NormalTok{p1)}
    
\NormalTok{    Nsub3 <-}\StringTok{ }\KeywordTok{subset}\NormalTok{(data,data}\OperatorTok{$}\NormalTok{ProductType }\OperatorTok{==}\StringTok{ }\NormalTok{p2)}
    
\NormalTok{    Nsub4 <-}\StringTok{ }\KeywordTok{subeset}\NormalTok{(data,data}\OperatorTok{$}\NormalTok{ProductType }\OperatorTok{==}\StringTok{ }\NormalTok{p3)}
    
\NormalTok{    Nsub4 <-}\StringTok{ }\KeywordTok{rbind}\NormalTok{(Nsub,Nsub2,Nsub3,Nsub4)}
    
    \KeywordTok{return}\NormalTok{(Nsub4)}
\NormalTok{  \}}
  
  \ControlFlowTok{else}\NormalTok{\{}
    
\NormalTok{    Nsub <-}\StringTok{ }\KeywordTok{subset}\NormalTok{(data, data}\OperatorTok{$}\NormalTok{ProductType }\OperatorTok{==}\StringTok{ }\NormalTok{p)}
    
\NormalTok{    Nsub2 <-}\KeywordTok{subset}\NormalTok{(data, data}\OperatorTok{$}\NormalTok{ProductType }\OperatorTok{==}\StringTok{ }\NormalTok{p1)}
    
\NormalTok{    Nsub3 <-}\StringTok{ }\KeywordTok{subset}\NormalTok{(data,data}\OperatorTok{$}\NormalTok{ProductType }\OperatorTok{==}\StringTok{ }\NormalTok{p2)}
    
\NormalTok{    Nsub4 <-}\StringTok{ }\KeywordTok{subeset}\NormalTok{(data,data}\OperatorTok{$}\NormalTok{ProductType }\OperatorTok{==}\StringTok{ }\NormalTok{p3)}
    
\NormalTok{    Nsub5 <-}\StringTok{ }\KeywordTok{subset}\NormalTok{(data,data}\OperatorTok{$}\NormalTok{ProductType }\OperatorTok{==}\StringTok{ }\NormalTok{p4)}
    
\NormalTok{    Nsub5 <-}\StringTok{ }\KeywordTok{rbind}\NormalTok{(Nsub,Nsub2,Nsub3,Nsub4,Nsub5)}
    
    \KeywordTok{return}\NormalTok{(Nsub5)}
\NormalTok{  \} }
\NormalTok{\}}

\NormalTok{#### I know it's not the most pretty or effective way to do this, but it works.}
\end{Highlighting}
\end{Shaded}

\subsubsection{Correlation Matrix :}\label{correlation-matrix}

\\

\begin{Shaded}
\begin{Highlighting}[]
\NormalTok{EP <-}\StringTok{ }\KeywordTok{PPfunction}\NormalTok{(EP)}
\NormalTok{EP <-}\StringTok{ }\KeywordTok{RmOut}\NormalTok{(EP,Volume)}

\NormalTok{corr_all<-}\KeywordTok{cor}\NormalTok{(EP)}



\NormalTok{corrplot}\OperatorTok{::}\StringTok{ }\KeywordTok{corrplot}\NormalTok{(corr_all,}\DataTypeTok{type=}\StringTok{"upper"}\NormalTok{,}\DataTypeTok{tl.pos=}\StringTok{"td"}\NormalTok{,}\DataTypeTok{method=}\StringTok{"circle"}\NormalTok{,}\DataTypeTok{tl.cex =} \FloatTok{0.5}\NormalTok{,}\DataTypeTok{tl.col=}\StringTok{'black'}\NormalTok{,}\DataTypeTok{diag=}\OtherTok{FALSE}\NormalTok{)}
\end{Highlighting}
\end{Shaded}

\includegraphics{Report_files/figure-latex/unnamed-chunk-4-1.pdf}

This information does not differ from the module 1's counterpart, it's
obvius because we are using the same data set.

\\

We trained a random forest followed by the use of varImp() function that
assess the importance of each variables (without the ones we took out by
looking at the correlation matrix). But first we need to create Test and
Training sets, we also came up with a simple function to automate the
processs.

\paragraph{Train and Test Set
function}\label{train-and-test-set-function}

\begin{Shaded}
\begin{Highlighting}[]
\NormalTok{TrainAndTestSets <-}\StringTok{ }\ControlFlowTok{function}\NormalTok{(label,p,data,seed)\{}
  \KeywordTok{set.seed}\NormalTok{(seed)}
  
\NormalTok{  inTrain <-}\StringTok{ }\KeywordTok{createDataPartition}\NormalTok{(}\DataTypeTok{y=}\NormalTok{ label, }\DataTypeTok{p =}\NormalTok{ p , }\DataTypeTok{list =} \OtherTok{FALSE}\NormalTok{)}
\NormalTok{  training <-}\StringTok{ }\NormalTok{data[inTrain,]}
\NormalTok{  testing <-}\StringTok{ }\NormalTok{data[}\OperatorTok{-}\NormalTok{inTrain,]}
  
  
  \KeywordTok{list}\NormalTok{(}\DataTypeTok{trainingSet=}\NormalTok{training,}\DataTypeTok{testingSet =}\NormalTok{ testing)}
  
\NormalTok{\}}
\end{Highlighting}
\end{Shaded}

\begin{Shaded}
\begin{Highlighting}[]
\NormalTok{EP <-}\StringTok{ }\NormalTok{EP[,}\KeywordTok{c}\NormalTok{(}\DecValTok{1}\NormalTok{,}\DecValTok{2}\NormalTok{,}\DecValTok{3}\NormalTok{,}\DecValTok{4}\NormalTok{,}\DecValTok{5}\NormalTok{,}\DecValTok{6}\NormalTok{,}\DecValTok{7}\NormalTok{,}\DecValTok{8}\NormalTok{,}\DecValTok{9}\NormalTok{,}\DecValTok{10}\NormalTok{,}\DecValTok{11}\NormalTok{,}\DecValTok{12}\NormalTok{,}\DecValTok{13}\NormalTok{,}\DecValTok{14}\NormalTok{,}\DecValTok{16}\NormalTok{,}\DecValTok{18}\NormalTok{,}\DecValTok{20}\NormalTok{,}\DecValTok{21}\NormalTok{,}\DecValTok{22}\NormalTok{,}\DecValTok{23}\NormalTok{,}\DecValTok{24}\NormalTok{,}\DecValTok{25}\NormalTok{,}\DecValTok{26}\NormalTok{,}\DecValTok{27}\NormalTok{,}\DecValTok{28}\NormalTok{)]}

\NormalTok{List <-}\StringTok{ }\KeywordTok{TrainAndTestSets}\NormalTok{(EP}\OperatorTok{$}\NormalTok{Volume,}\FloatTok{0.75}\NormalTok{,EP,}\DecValTok{123}\NormalTok{)}
\end{Highlighting}
\end{Shaded}

\begin{Shaded}
\begin{Highlighting}[]
\NormalTok{ fitcontrol <-}\StringTok{  }\KeywordTok{trainControl}\NormalTok{(}\DataTypeTok{method =} \StringTok{"repeatedcv"}\NormalTok{, }\DataTypeTok{repeats =} \DecValTok{4}\NormalTok{)}

\NormalTok{  Model <-}\StringTok{ }\KeywordTok{train}\NormalTok{(Volume}\OperatorTok{~}\NormalTok{., }\DataTypeTok{data =}\NormalTok{ EP,}\DataTypeTok{method =} \StringTok{"rf"}\NormalTok{, }\DataTypeTok{trcontrol =}\NormalTok{ fitcontrol , }\DataTypeTok{tunelenght =} \DecValTok{5}
\NormalTok{                 , }\DataTypeTok{preProcess =} \KeywordTok{c}\NormalTok{(}\StringTok{"center"}\NormalTok{, }\StringTok{"scale"}\NormalTok{),}\DataTypeTok{importance=}\NormalTok{T)  }
  
  \KeywordTok{varImp}\NormalTok{(Model)}
\end{Highlighting}
\end{Shaded}

\begin{verbatim}
## rf variable importance
## 
##   only 20 most important variables shown (out of 24)
## 
##                              Overall
## PositiveServiceReview        100.000
## x4StarReviews                 46.607
## ProductWidth                  11.786
## ShippingWeight                10.171
## Price                          9.865
## x2StarReviews                  9.387
## ProductType.ExtendedWarranty   8.821
## ProductType.Printer            7.122
## ProfitMargin                   6.530
## ProductDepth                   6.083
## ProductType.Tablet             6.030
## ProductType.Software           5.748
## ProductHeight                  4.535
## ProductType.GameConsole        3.915
## Recommendproduct               3.785
## ProductType.Accessories        3.465
## ProductType.Smartphone         3.014
## ProductType.PC                 2.530
## ProductType.Display            2.499
## NegativeServiceReview          2.344
\end{verbatim}

So the only variables with a significant impact are only
PostiveServiceReview and x4StarReviews.

\subsection{Models and Predictions}\label{models-and-predictions}

\subsubsection{Training Function}\label{training-function}

 I created a function that trains every different model, by user
specification

\begin{Shaded}
\begin{Highlighting}[]
\NormalTok{EP <-}\StringTok{ }\KeywordTok{read.csv}\NormalTok{( }\DataTypeTok{file =}\StringTok{"/home/zordo/Documents/Ubiqum/R-M2Task3/data/Epa.csv"}\NormalTok{ , }\DataTypeTok{header =} \OtherTok{TRUE}\NormalTok{ , }\DataTypeTok{sep =} \StringTok{','}\NormalTok{)}

\NormalTok{EP <-}\StringTok{ }\NormalTok{EP[,}\KeywordTok{c}\NormalTok{(}\DecValTok{1}\NormalTok{,}\DecValTok{5}\NormalTok{,}\DecValTok{9}\NormalTok{,}\DecValTok{18}\NormalTok{)]}

\NormalTok{EP <-}\StringTok{ }\KeywordTok{PPfunction}\NormalTok{(EP)}

\NormalTok{EP <-}\StringTok{ }\KeywordTok{RmOut}\NormalTok{(EP)}

\NormalTok{List <-}\StringTok{ }\KeywordTok{TrainAndTestSets}\NormalTok{(EP}\OperatorTok{$}\NormalTok{Volume,}\FloatTok{0.75}\NormalTok{,EP,}\DecValTok{123}\NormalTok{)}


\NormalTok{#### Random Forest ####}
\NormalTok{ModelRandomForest <-}\StringTok{ }\KeywordTok{TrainingFunction}\NormalTok{(}\StringTok{"rf"}\NormalTok{,Volume}\OperatorTok{~}\NormalTok{.,List}\OperatorTok{$}\NormalTok{trainingSet,}\DecValTok{5}\NormalTok{)}

\NormalTok{PredictionRandomForest <-}\StringTok{ }\KeywordTok{predict}\NormalTok{(ModelRandomForest,List}\OperatorTok{$}\NormalTok{testingSet)}

\NormalTok{TestResultsRF <-}\StringTok{ }\KeywordTok{postResample}\NormalTok{(PredictionRandomForest,List}\OperatorTok{$}\NormalTok{testingSet}\OperatorTok{$}\NormalTok{Volume)}

\NormalTok{#### SVM ####}

\NormalTok{svm.model <-}\StringTok{ }\KeywordTok{TrainingFunction}\NormalTok{(}\StringTok{"svm"}\NormalTok{,Volume}\OperatorTok{~}\NormalTok{.,List}\OperatorTok{$}\NormalTok{trainingSet,}\DecValTok{5}\NormalTok{,}\DecValTok{10000000}\NormalTok{,}\FloatTok{0.0000001}\NormalTok{)  }

\NormalTok{svm.pred <-}\StringTok{ }\KeywordTok{predict}\NormalTok{(svm.model,List}\OperatorTok{$}\NormalTok{testingSet)}

\NormalTok{TestResultsSVM <-}\StringTok{ }\KeywordTok{postResample}\NormalTok{(svm.pred,List}\OperatorTok{$}\NormalTok{testingSet}\OperatorTok{$}\NormalTok{Volume)}


\NormalTok{#### knn ####}

\NormalTok{ KNN <-}\StringTok{ }\KeywordTok{TrainingFunction}\NormalTok{(}\StringTok{"knn"}\NormalTok{,Volume}\OperatorTok{~}\NormalTok{.,List}\OperatorTok{$}\NormalTok{trainingSet,}\DecValTok{30}\NormalTok{)}

\NormalTok{KnnPrediction <-}\StringTok{ }\KeywordTok{predict}\NormalTok{(KNN,List}\OperatorTok{$}\NormalTok{testingSet)}

\NormalTok{TestResultsKNN <-}\KeywordTok{postResample}\NormalTok{(KnnPrediction,List}\OperatorTok{$}\NormalTok{testingSet}\OperatorTok{$}\NormalTok{Volume)}

\NormalTok{####}


\NormalTok{AllTestResults <-}\StringTok{ }\KeywordTok{cbind}\NormalTok{(TestResultsKNN,TestResultsRF,TestResultsSVM)}

\NormalTok{AllTestResults    }
\end{Highlighting}
\end{Shaded}

\begin{verbatim}
##          TestResultsKNN TestResultsRF TestResultsSVM
## RMSE        429.1288137   271.9646084    788.7251446
## Rsquared      0.6828584     0.8447768      0.2698622
## MAE         250.9259259   156.6409058    553.4143148
\end{verbatim}

And then did another one t train the three models at the same time with
a for loop

\begin{Shaded}
\begin{Highlighting}[]
\NormalTok{TrainAll3Models <-}\StringTok{ }\ControlFlowTok{function}\NormalTok{ (formula,data)}
\NormalTok{  \{}

\NormalTok{  Model <-}\StringTok{ }\KeywordTok{vector}\NormalTok{(}\DataTypeTok{mode=}\StringTok{"list"}\NormalTok{, }\DataTypeTok{length=}\KeywordTok{length}\NormalTok{(methods))}

\NormalTok{      methods <-}\StringTok{ }\KeywordTok{c}\NormalTok{(}\StringTok{"rf"}\NormalTok{,}\StringTok{"svm"}\NormalTok{,}\StringTok{"knn"}\NormalTok{)}
        
      \ControlFlowTok{for}\NormalTok{(i }\ControlFlowTok{in} \DecValTok{1}\OperatorTok{:}\KeywordTok{length}\NormalTok{(methods))}
\NormalTok{          \{    }
      
\NormalTok{            Model[[i]] <-}\StringTok{ }\KeywordTok{TrainingFunction}\NormalTok{(methods[i],formula,data,}\DecValTok{5}\NormalTok{)}

               

\NormalTok{      \}}
\NormalTok{    Model}
\NormalTok{\}}
\end{Highlighting}
\end{Shaded}

 I didn't use this function that much since the mentors showed us
another way of training without any function, and it's much easier and
cleaner.

\begin{Shaded}
\begin{Highlighting}[]
\NormalTok{a <-}\StringTok{ }\KeywordTok{c}\NormalTok{(}\StringTok{"Volume ~ x4StarReviews"}\NormalTok{,}\StringTok{"Volume ~."}\NormalTok{,}\StringTok{"Volume ~ PositiveServiceReview"}\NormalTok{)}
\NormalTok{b <-}\StringTok{ }\KeywordTok{c}\NormalTok{(}\StringTok{"lm"}\NormalTok{,}\StringTok{"rf"}\NormalTok{, }\StringTok{"knn"}\NormalTok{,}\StringTok{"svmLinear"}\NormalTok{)}
\NormalTok{compare_var_mod <-}\StringTok{ }\KeywordTok{c}\NormalTok{()}

\ControlFlowTok{for}\NormalTok{ ( i   }\ControlFlowTok{in}\NormalTok{ a) \{}
  \ControlFlowTok{for}\NormalTok{ (j }\ControlFlowTok{in}\NormalTok{ b) \{}
    
\NormalTok{    model <-}\StringTok{ }\KeywordTok{train}\NormalTok{(}\KeywordTok{formula}\NormalTok{(i), }\DataTypeTok{data =}\NormalTok{ List}\OperatorTok{$}\NormalTok{trainingSet, }\DataTypeTok{method =}\NormalTok{ b,}\DataTypeTok{trainControl=}\KeywordTok{trainControl}\NormalTok{(}\DataTypeTok{method =} \StringTok{"repeatedcv"}\NormalTok{, }\DataTypeTok{repeats =} \DecValTok{4}\NormalTok{))}
    
\NormalTok{    pred <-}\StringTok{ }\KeywordTok{predict}\NormalTok{(model, }\DataTypeTok{newdata =}\NormalTok{ List}\OperatorTok{$}\NormalTok{testingSet)}
    
\NormalTok{    pred_metric <-}\StringTok{ }\KeywordTok{postResample}\NormalTok{(List}\OperatorTok{$}\NormalTok{testingSet}\OperatorTok{$}\NormalTok{Volume, pred)}
    
\NormalTok{    compare_var_mod <-}\StringTok{ }\KeywordTok{cbind}\NormalTok{(compare_var_mod , pred_metric)}
    
\NormalTok{  \}}
  
\NormalTok{\}}
\NormalTok{      compare_var_mod}
\end{Highlighting}
\end{Shaded}

\begin{verbatim}
##          pred_metric pred_metric pred_metric pred_metric pred_metric
## RMSE     459.4407974 459.4407974 459.4407974 459.4407974 482.8296811
## Rsquared   0.5286655   0.5286655   0.5286655   0.5286655   0.5398693
## MAE      338.0400480 338.0400480 338.0400480 338.0400480 254.6791507
##          pred_metric pred_metric pred_metric pred_metric pred_metric
## RMSE     482.8296811 482.8296811 482.8296811 568.6757233 568.6757233
## Rsquared   0.5398693   0.5398693   0.5398693   0.2851607   0.2851607
## MAE      254.6791507 254.6791507 254.6791507 436.9034926 436.9034926
##          pred_metric pred_metric
## RMSE     568.6757233 568.6757233
## Rsquared   0.2851607   0.2851607
## MAE      436.9034926 436.9034926
\end{verbatim}

\begin{Shaded}
\begin{Highlighting}[]
\NormalTok{names_var <-}\StringTok{ }\KeywordTok{c}\NormalTok{()}
\ControlFlowTok{for}\NormalTok{ (i }\ControlFlowTok{in}\NormalTok{ a) \{}
  \ControlFlowTok{for}\NormalTok{(j }\ControlFlowTok{in}\NormalTok{ b) \{}
\NormalTok{    names_var <-}\StringTok{ }\KeywordTok{append}\NormalTok{(names_var,}\KeywordTok{paste}\NormalTok{(i,j))}
\NormalTok{  \}}
\NormalTok{\}}


\NormalTok{names_var}
\end{Highlighting}
\end{Shaded}

\begin{verbatim}
##  [1] "Volume ~ x4StarReviews lm"               
##  [2] "Volume ~ x4StarReviews rf"               
##  [3] "Volume ~ x4StarReviews knn"              
##  [4] "Volume ~ x4StarReviews svmLinear"        
##  [5] "Volume ~. lm"                            
##  [6] "Volume ~. rf"                            
##  [7] "Volume ~. knn"                           
##  [8] "Volume ~. svmLinear"                     
##  [9] "Volume ~ PositiveServiceReview lm"       
## [10] "Volume ~ PositiveServiceReview rf"       
## [11] "Volume ~ PositiveServiceReview knn"      
## [12] "Volume ~ PositiveServiceReview svmLinear"
\end{verbatim}

\begin{Shaded}
\begin{Highlighting}[]
\KeywordTok{colnames}\NormalTok{(compare_var_mod) <-}\StringTok{ }\NormalTok{names_var}

\NormalTok{compare_var_mod}
\end{Highlighting}
\end{Shaded}

\begin{verbatim}
##          Volume ~ x4StarReviews lm Volume ~ x4StarReviews rf
## RMSE                   459.4407974               459.4407974
## Rsquared                 0.5286655                 0.5286655
## MAE                    338.0400480               338.0400480
##          Volume ~ x4StarReviews knn Volume ~ x4StarReviews svmLinear
## RMSE                    459.4407974                      459.4407974
## Rsquared                  0.5286655                        0.5286655
## MAE                     338.0400480                      338.0400480
##          Volume ~. lm Volume ~. rf Volume ~. knn Volume ~. svmLinear
## RMSE      482.8296811  482.8296811   482.8296811         482.8296811
## Rsquared    0.5398693    0.5398693     0.5398693           0.5398693
## MAE       254.6791507  254.6791507   254.6791507         254.6791507
##          Volume ~ PositiveServiceReview lm
## RMSE                           568.6757233
## Rsquared                         0.2851607
## MAE                            436.9034926
##          Volume ~ PositiveServiceReview rf
## RMSE                           568.6757233
## Rsquared                         0.2851607
## MAE                            436.9034926
##          Volume ~ PositiveServiceReview knn
## RMSE                            568.6757233
## Rsquared                          0.2851607
## MAE                             436.9034926
##          Volume ~ PositiveServiceReview svmLinear
## RMSE                                  568.6757233
## Rsquared                                0.2851607
## MAE                                   436.9034926
\end{verbatim}

\begin{Shaded}
\begin{Highlighting}[]
\NormalTok{compare_var_mod_melt <-}\StringTok{ }\KeywordTok{melt}\NormalTok{(compare_var_mod, }\DataTypeTok{varnames =} \KeywordTok{c}\NormalTok{(}\StringTok{"metric"}\NormalTok{, }\StringTok{"model"}\NormalTok{))}
\NormalTok{compare_var_mod_melt <-}\StringTok{ }\KeywordTok{as.data.frame}\NormalTok{(compare_var_mod_melt)}
\NormalTok{compare_var_mod_melt}
\end{Highlighting}
\end{Shaded}

\begin{verbatim}
##      metric                                    model       value
## 1      RMSE                Volume ~ x4StarReviews lm 459.4407974
## 2  Rsquared                Volume ~ x4StarReviews lm   0.5286655
## 3       MAE                Volume ~ x4StarReviews lm 338.0400480
## 4      RMSE                Volume ~ x4StarReviews rf 459.4407974
## 5  Rsquared                Volume ~ x4StarReviews rf   0.5286655
## 6       MAE                Volume ~ x4StarReviews rf 338.0400480
## 7      RMSE               Volume ~ x4StarReviews knn 459.4407974
## 8  Rsquared               Volume ~ x4StarReviews knn   0.5286655
## 9       MAE               Volume ~ x4StarReviews knn 338.0400480
## 10     RMSE         Volume ~ x4StarReviews svmLinear 459.4407974
## 11 Rsquared         Volume ~ x4StarReviews svmLinear   0.5286655
## 12      MAE         Volume ~ x4StarReviews svmLinear 338.0400480
## 13     RMSE                             Volume ~. lm 482.8296811
## 14 Rsquared                             Volume ~. lm   0.5398693
## 15      MAE                             Volume ~. lm 254.6791507
## 16     RMSE                             Volume ~. rf 482.8296811
## 17 Rsquared                             Volume ~. rf   0.5398693
## 18      MAE                             Volume ~. rf 254.6791507
## 19     RMSE                            Volume ~. knn 482.8296811
## 20 Rsquared                            Volume ~. knn   0.5398693
## 21      MAE                            Volume ~. knn 254.6791507
## 22     RMSE                      Volume ~. svmLinear 482.8296811
## 23 Rsquared                      Volume ~. svmLinear   0.5398693
## 24      MAE                      Volume ~. svmLinear 254.6791507
## 25     RMSE        Volume ~ PositiveServiceReview lm 568.6757233
## 26 Rsquared        Volume ~ PositiveServiceReview lm   0.2851607
## 27      MAE        Volume ~ PositiveServiceReview lm 436.9034926
## 28     RMSE        Volume ~ PositiveServiceReview rf 568.6757233
## 29 Rsquared        Volume ~ PositiveServiceReview rf   0.2851607
## 30      MAE        Volume ~ PositiveServiceReview rf 436.9034926
## 31     RMSE       Volume ~ PositiveServiceReview knn 568.6757233
## 32 Rsquared       Volume ~ PositiveServiceReview knn   0.2851607
## 33      MAE       Volume ~ PositiveServiceReview knn 436.9034926
## 34     RMSE Volume ~ PositiveServiceReview svmLinear 568.6757233
## 35 Rsquared Volume ~ PositiveServiceReview svmLinear   0.2851607
## 36      MAE Volume ~ PositiveServiceReview svmLinear 436.9034926
\end{verbatim}

\begin{Shaded}
\begin{Highlighting}[]
\KeywordTok{ggplot}\NormalTok{(compare_var_mod_melt, }\KeywordTok{aes}\NormalTok{(}\DataTypeTok{x=}\NormalTok{model,}\DataTypeTok{y=}\NormalTok{value)) }\OperatorTok{+}\StringTok{ }\KeywordTok{geom_col}\NormalTok{() }\OperatorTok{+}\StringTok{ }\KeywordTok{facet_grid}\NormalTok{(metric}\OperatorTok{~}\NormalTok{., }\DataTypeTok{scales=}\StringTok{"free"}\NormalTok{) }\OperatorTok{+}\KeywordTok{theme}\NormalTok{(}\DataTypeTok{axis.text=}\KeywordTok{element_text}\NormalTok{(}\DataTypeTok{size=}\DecValTok{3}\NormalTok{),}
        \DataTypeTok{axis.title=}\KeywordTok{element_text}\NormalTok{(}\DataTypeTok{size=}\DecValTok{14}\NormalTok{,}\DataTypeTok{face=}\StringTok{"bold"}\NormalTok{))}
\end{Highlighting}
\end{Shaded}

\includegraphics{Report_files/figure-latex/unnamed-chunk-11-1.pdf} I
only used RF, and KNN because from past results the SVM did not look
like a good fit.

\subsubsection{Error Analysis}\label{error-analysis}

\begin{Shaded}
\begin{Highlighting}[]
\NormalTok{ABSrf <-}\StringTok{ }\NormalTok{(List}\OperatorTok{$}\NormalTok{testingSet}\OperatorTok{$}\NormalTok{Volume }\OperatorTok{-}\StringTok{ }\NormalTok{PredictionRandomForest)}

\NormalTok{RLTrf <-}\StringTok{  }\NormalTok{(ABSrf }\OperatorTok{/}\StringTok{ }\NormalTok{List}\OperatorTok{$}\NormalTok{testingSet}\OperatorTok{$}\NormalTok{Volume)}

\NormalTok{ABSsvm <-}\StringTok{ }\NormalTok{(List}\OperatorTok{$}\NormalTok{testingSet}\OperatorTok{$}\NormalTok{Volume }\OperatorTok{-}\StringTok{ }\NormalTok{svm.pred)}

\NormalTok{RLTsvm <-}\StringTok{ }\NormalTok{(ABSsvm }\OperatorTok{/}\StringTok{ }\NormalTok{List}\OperatorTok{$}\NormalTok{testingSet}\OperatorTok{$}\NormalTok{Volume)}



\NormalTok{Absknn <-}\StringTok{ }\NormalTok{(List}\OperatorTok{$}\NormalTok{testingSet}\OperatorTok{$}\NormalTok{Volume }\OperatorTok{-}\StringTok{ }\NormalTok{KnnPrediction)}

\NormalTok{RLTknn <-}\StringTok{  }\NormalTok{(Absknn }\OperatorTok{/}\StringTok{ }\NormalTok{List}\OperatorTok{$}\NormalTok{testingSet}\OperatorTok{$}\NormalTok{Volume)}
 \CommentTok{#abline(0, 0)                  # the horizon}


\NormalTok{Lol <-}\StringTok{ }\KeywordTok{cbind}\NormalTok{(List}\OperatorTok{$}\NormalTok{testingSet,ABSrf)}
\end{Highlighting}
\end{Shaded}

 Random Forest Residuals

\begin{Shaded}
\begin{Highlighting}[]
\KeywordTok{ggplot}\NormalTok{(Lol,}
      \KeywordTok{aes}\NormalTok{(Lol}\OperatorTok{$}\NormalTok{Volume,ABSrf))}\OperatorTok{+}
\StringTok{ }\KeywordTok{geom_point}\NormalTok{(}\DataTypeTok{color=}\StringTok{"red"}\NormalTok{)}\OperatorTok{+}
\StringTok{ }\KeywordTok{geom_smooth}\NormalTok{()}
\end{Highlighting}
\end{Shaded}

\begin{verbatim}
## `geom_smooth()` using method = 'loess' and formula 'y ~ x'
\end{verbatim}

\includegraphics{Report_files/figure-latex/unnamed-chunk-13-1.pdf}

\begin{Shaded}
\begin{Highlighting}[]
\KeywordTok{ggplot}\NormalTok{(Lol,}
      \KeywordTok{aes}\NormalTok{(Lol}\OperatorTok{$}\NormalTok{Volume,RLTrf))}\OperatorTok{+}
\StringTok{ }\KeywordTok{geom_point}\NormalTok{(}\DataTypeTok{color=}\StringTok{"red"}\NormalTok{)}\OperatorTok{+}
\StringTok{ }\KeywordTok{geom_smooth}\NormalTok{()}
\end{Highlighting}
\end{Shaded}

\begin{verbatim}
## `geom_smooth()` using method = 'loess' and formula 'y ~ x'
\end{verbatim}

\includegraphics{Report_files/figure-latex/unnamed-chunk-13-2.pdf} Svm
Residuals

\begin{Shaded}
\begin{Highlighting}[]
\KeywordTok{ggplot}\NormalTok{(Lol,}
      \KeywordTok{aes}\NormalTok{(Lol}\OperatorTok{$}\NormalTok{Volume,ABSsvm))}\OperatorTok{+}
\StringTok{ }\KeywordTok{geom_point}\NormalTok{(}\DataTypeTok{color=}\StringTok{"red"}\NormalTok{)}\OperatorTok{+}
\StringTok{ }\KeywordTok{geom_smooth}\NormalTok{()}
\end{Highlighting}
\end{Shaded}

\begin{verbatim}
## `geom_smooth()` using method = 'loess' and formula 'y ~ x'
\end{verbatim}

\includegraphics{Report_files/figure-latex/unnamed-chunk-14-1.pdf}

\begin{Shaded}
\begin{Highlighting}[]
\KeywordTok{ggplot}\NormalTok{(Lol,}
      \KeywordTok{aes}\NormalTok{(Lol}\OperatorTok{$}\NormalTok{Volume,RLTsvm))}\OperatorTok{+}
\StringTok{ }\KeywordTok{geom_point}\NormalTok{(}\DataTypeTok{color=}\StringTok{"red"}\NormalTok{)}\OperatorTok{+}
\StringTok{ }\KeywordTok{geom_smooth}\NormalTok{()}
\end{Highlighting}
\end{Shaded}

\begin{verbatim}
## `geom_smooth()` using method = 'loess' and formula 'y ~ x'
\end{verbatim}

\includegraphics{Report_files/figure-latex/unnamed-chunk-14-2.pdf}

\begin{Shaded}
\begin{Highlighting}[]
\KeywordTok{ggplot}\NormalTok{(Lol,}
      \KeywordTok{aes}\NormalTok{(Lol}\OperatorTok{$}\NormalTok{Volume,Absknn))}\OperatorTok{+}
\StringTok{ }\KeywordTok{geom_point}\NormalTok{(}\DataTypeTok{color=}\StringTok{"red"}\NormalTok{)}\OperatorTok{+}
\StringTok{ }\KeywordTok{geom_smooth}\NormalTok{()}
\end{Highlighting}
\end{Shaded}

\begin{verbatim}
## `geom_smooth()` using method = 'loess' and formula 'y ~ x'
\end{verbatim}

\includegraphics{Report_files/figure-latex/unnamed-chunk-15-1.pdf}

\begin{Shaded}
\begin{Highlighting}[]
\KeywordTok{ggplot}\NormalTok{(Lol,}
      \KeywordTok{aes}\NormalTok{(Lol}\OperatorTok{$}\NormalTok{Volume,RLTknn))}\OperatorTok{+}
\StringTok{ }\KeywordTok{geom_point}\NormalTok{(}\DataTypeTok{color=}\StringTok{"red"}\NormalTok{)}\OperatorTok{+}
\StringTok{ }\KeywordTok{geom_smooth}\NormalTok{()}
\end{Highlighting}
\end{Shaded}

\begin{verbatim}
## `geom_smooth()` using method = 'loess' and formula 'y ~ x'
\end{verbatim}

\includegraphics{Report_files/figure-latex/unnamed-chunk-15-2.pdf}

 Let's now apply the current models into the new product list and make a
top 5 for most probably sold products in volume.

\subsubsection{Prediction}\label{prediction}

 The random forest was the one who gave us the best results, with both
variables (ProductServiceReview,x4StarReviews).

\begin{Shaded}
\begin{Highlighting}[]
\NormalTok{NP <-}\StringTok{ }\KeywordTok{read.csv}\NormalTok{(}\DataTypeTok{file =} \StringTok{"/home/zordo/Documents/Ubiqum/R-M2Task3/data/Npa.csv"}\NormalTok{, }\DataTypeTok{header =} \OtherTok{TRUE}\NormalTok{ , }\DataTypeTok{sep =}\StringTok{','}\NormalTok{)}

\NormalTok{NP <-}\StringTok{ }\KeywordTok{PPfunction}\NormalTok{(NP)}

\NormalTok{NewProductsVolume <-}\StringTok{ }\KeywordTok{predict}\NormalTok{(ModelRandomForest,NP)}

\NormalTok{NP}\OperatorTok{$}\NormalTok{Volume<-NewProductsVolume}
\end{Highlighting}
\end{Shaded}

 \\
 \\
We got the volume, now we need to calculate the profit, to see which
products types we should invest on.

\textbf{Profit = profit margin * Volume}

 \\

\begin{Shaded}
\begin{Highlighting}[]
\NormalTok{  Profit <-}\StringTok{ }\NormalTok{NP}\OperatorTok{$}\NormalTok{ProfitMargin }\OperatorTok{*}\StringTok{ }\NormalTok{NP}\OperatorTok{$}\NormalTok{Volume}

\NormalTok{  NP <-}\StringTok{ }\KeywordTok{cbind}\NormalTok{(NP,Profit)  }

  
\NormalTok{Top5 <-}\StringTok{  }\KeywordTok{top_n}\NormalTok{(NP, }\DecValTok{5}\NormalTok{, Profit)}
  
\NormalTok{Top5 <-}\StringTok{ }\KeywordTok{cbind}\NormalTok{ (Top5,}\KeywordTok{sort}\NormalTok{(Top5}\OperatorTok{$}\NormalTok{Profit))}
 
\NormalTok{Top5}
\end{Highlighting}
\end{Shaded}

\begin{verbatim}
##   ProductType.Accessories ProductType.Display ProductType.ExtendedWarranty
## 1                       0                   0                            0
## 2                       0                   0                            0
## 3                       0                   0                            0
## 4                       0                   0                            0
## 5                       0                   0                            0
##   ProductType.GameConsole ProductType.Laptop ProductType.Netbook
## 1                       0                  0                   1
## 2                       0                  0                   0
## 3                       0                  0                   0
## 4                       1                  0                   0
## 5                       1                  0                   0
##   ProductType.PC ProductType.Printer ProductType.PrinterSupplies
## 1              0                   0                           0
## 2              0                   0                           0
## 3              0                   0                           0
## 4              0                   0                           0
## 5              0                   0                           0
##   ProductType.Smartphone ProductType.Software ProductType.Tablet
## 1                      0                    0                  0
## 2                      0                    0                  1
## 3                      0                    0                  1
## 4                      0                    0                  0
## 5                      0                    0                  0
##   ProductNum  Price x5StarReviews x4StarReviews x3StarReviews
## 1        180 329.00           312           112            28
## 2        186 629.00           296            66            30
## 3        187 199.00           943           437           224
## 4        199 249.99           462            97            25
## 5        307 425.00          1525           252            99
##   x2StarReviews x1StarReviews PositiveServiceReview NegativeServiceReview
## 1            31            47                    28                    16
## 2            21            36                    28                     9
## 3           160           247                    90                    23
## 4            17            58                    32                    12
## 5            56            45                    59                    13
##   Recommendproduct BestSellersRank ShippingWeight ProductDepth
## 1              0.7            2699            4.6        10.17
## 2              0.8              34            3.0         7.31
## 3              0.8               1            0.9         5.40
## 4              0.8             115            8.4         6.20
## 5              0.9             215           20.0         8.50
##   ProductWidth ProductHeight ProfitMargin   Volume   Profit
## 1         7.28          0.95         0.09 1386.986 124.8287
## 2         9.50          0.37         0.10 1371.081 137.1081
## 3         7.60          0.40         0.20 1835.990 367.1979
## 4        13.20         13.20         0.09 1513.556 136.2200
## 5         6.00          1.75         0.18 1843.867 331.8960
##   sort(Top5$Profit)
## 1          124.8287
## 2          136.2200
## 3          137.1081
## 4          331.8960
## 5          367.1979
\end{verbatim}

 \\
 Our top 5 most profitable product types are \\
 \\

\begin{longtable}[]{@{}lcr@{}}
\toprule
Number & Volume & Profit\tabularnewline
\midrule
\endhead
Game Console & 1735.3872 & 347.07744\tabularnewline
Game Console & 1735.3872 & 312.36970\tabularnewline
Tablet & 1319.6391 & 118.76752\tabularnewline
Tablet & 1211.3776 & 109.02398\tabularnewline
NoteBook & 877.6156 & 87.76156\tabularnewline
\bottomrule
\end{longtable}

\\
 \\

\subsection{Conclusion}\label{conclusion}

 \\

All three models used are non-parametric, but before explaining what a
non-parametric model is, I would like to explain what parametric models
are. Parametric models are algorithms that simplify the function to a
known form, and no matter how much data that's fed to the algorithm, the
model won't change the quantity of parameters needed.All you need to
know for predicting a future data balue from the current state of the
model are his parameters,EG : Linear regression with on variable, you
have two parameters (coefficient and intercept).Knowing this parameters
will enable you to predict new values. In a mathematical way : \\
\textbf{Yi = B0 + B1X1 + B2X2 + \ldots{} + ei} Non-parametric models do
not make any fixed assumptions about the form of the mapping, they are
free to learn from the training data. The parameters are usually said to
be infinite in dimension and so can express the characteristics in data
much better than parametric models. E.g : KNN, makes predictions based
on the k most similar training patterns for a new data instance. The
method does not assume anything about the form of the mapping function
other than patterns that are close are likely have a similar output
variable. In mathematical form \textbf{Yi=F(xi) + ei} Where F can be any
function,the data will decide what the function looks like.It will not
provide the analytical expression but it will give you its graph given
your data set. I find this picture very helpfull in understanding how
both types of models work, summing up in easy non- technical language,
parametric models follow an equation while non-parametric models follow
the data. One of the requirements to use a non-parametric model is to
have a big set of data, we have 80 rows, 78 after cutting outliers and
only 60 of them are for training.This is not a big set of data at all, I
think it actually couldn't be any smaller than this. Also they are good
methods when we don't have any prior knowledge and not worry about
choosing the right features. So using this models is not efficient at
all, if we are using algorithms that ``follow the data'' and we have
almost no data, it's obvious why this approach is not efficient and why
the errors are so big.


\end{document}
